\documentclass{scrartcl}
\title{\rmfamily Software Engineering -- Blatt 9}
\author{Rasmus Diederichsen \and Felix Breuninger\and 
   \texttt{\{rdiederichse, fbreunin\}@uos.de}
}
\date{\today}
\usepackage[ngerman]{babel}
\usepackage[space]{grffile} % for spaces in includegraphics fnames
\usepackage{marvosym, microtype, textcomp, xifthen, multirow, booktabs, dingbat,
   titlesec, enumitem, fullpage, tikz, IEEEtrantools, array, amsmath, listings,
amssymb, graphicx, subcaption, lmodern,pgfplots}
\usepackage[pdftitle={Software Engineering -- Blatt 9}, 
   pdfauthor={Rasmus Diederichsen, Felix Breuninger}, 
   hyperfootnotes=true,
   colorlinks,
   bookmarksnumbered = true,
   linkcolor = lightgray,
   plainpages = false,
citecolor = lightgray]{hyperref}
\usepackage[utf8]{inputenc}
\usepackage[T1]{fontenc}
\usepackage[all]{hypcap}
\titleformat{\section}[hang]{\bf}{Aufgabe 9.\arabic{section}:}{1em}{}[]
\titleformat{\subsection}[hang]{\bf}{\hspace{1em}\alph{subsubsection})}{1em}{}[]

\lstset{
   frame=single,
   basicstyle=\ttfamily\small,
   frameround=tttt,
   backgroundcolor=\color{lightgray!10},
   keywordstyle=\color{teal}\textbf,
   stringstyle=\itshape,
   showstringspaces=false,
   language=[gnu] make,
   morecomment=[n]{$(}{)},
   commentstyle=\color{blue},
   title=\lstname
}
\usetikzlibrary{shapes,positioning,calc,decorations.text,graphs,arrows.meta}
\begin{document}

\fontfamily{ptm}\selectfont
\maketitle

\section{Aktivitätsdiagramm}

Der Koffer ist in \autoref{akti} ein einfaches Datum, da er nicht gefüllt oder entleert wird.
Er kann auch kein Datenspeicher sein, weil er selbst ja auch auf dem Datenfluss
fließt. Der Kofferraum hingegen ist wie der Briefkasten ein flüchtiger Speicher.
Wenn man Metaphysik außer Acht lässt, kann man sich relativ sicher sein, dass
der Koffer, den man reintut, der selbe ist, den man wieder herausholt. Zudem
dupliziert ein Kofferraum auch keine Gegenstände.

\begin{figure}
   {\centering      
      \includegraphics[width=\linewidth]{Reise.pdf}
      \caption{Aktivitätsdiagramm für Herrn Faber}
   \label{akti}}
\end{figure}

\end{document}
