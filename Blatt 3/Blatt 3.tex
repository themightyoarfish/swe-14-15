\documentclass{scrartcl}
\title{\rmfamily Software Engineering -- Blatt 3}
\author{Rasmus Diederichsen \and Felix Breuninger\and % ugly hack, why does \\ ... work no more?
   \texttt{\{rdiederichse, fbreunin\}@uos.de}
}
\date{\today}
\usepackage[ngerman]{babel}
            \usepackage[table]{xcolor}
\usepackage{microtype,
            textcomp,
            booktabs,
            dingbat,
            titlesec,
            enumitem,
            fullpage,
            tikz,
            IEEEtrantools,
            array,
            amsmath,
            amssymb,
            graphicx,
            subcaption,
            lmodern}
            \usepackage[pdftitle={Software Engineering -- Blatt 3}, 
       pdfauthor={Rasmus Diederichsen}, 
       hyperfootnotes=true,
       colorlinks,
       bookmarksnumbered = true,
       linkcolor = lightgray,
       plainpages = false,
       citecolor = lightgray]{hyperref}
\usepackage[utf8]{inputenc}
\usepackage[T1]{fontenc}
\usepackage[all]{hypcap}
% \renewcommand{\subsection}[1]{\noindent\bf Aufgabe \arabic{section}.\arabic{subsection}: #1}
\titleformat{\subsection}[hang]{\bf}{Aufgabe \arabic{section}.\arabic{subsection}:}{1em}{}[]
\titleformat{\subsubsection}[hang]{\bf}{\hspace{1em}\alph{subsubsection})}{1em}{}[]


\begin{document}

\fontfamily{ptm}\selectfont
\maketitle


\setcounter{section}{3}
\setcounter{subsection}{0}
\subsection{Projektstrukturplan (20 Punkte)}

Wir wählen einen Phasenorientieren PSP. Dies bietet sich an, da der Aufgabentext
diese schon explizit vorgibt, was uns deshalb Arbeit erspart.\leftthumbsup

\usetikzlibrary{trees,calc}
\begin{center}
\begin{tikzpicture}[
  work package/.style={draw,rectangle,text width=3cm},
  phase/.style={fill=gray!20,rounded corners=5pt,line width=1pt,text centered,font=\bf},
  grandchild/.style={grow=down,
  edge from parent path={(\tikzparentnode.west) -- ++(-1em,0) |- ($(\tikzparentnode.south west) + (-1em,0)$) |- (\tikzchildnode.west)}},
  first/.style={level distance=6ex},
  second/.style={level distance=14ex},
  third/.style={level distance=22ex},
  fourth/.style={level distance=30ex},
  fifth/.style={level distance=38ex},
  sixth/.style={level distance=46ex},
  seventh/.style={level distance=54ex},
  eighth/.style={level distance=62ex},
  level 1/.style={sibling distance=4cm,level distance=2cm}
  ]
  % \foreach \x / \y in {first/1,second/2,third/3,fourth/4,fifth/5,sixth/6,seventh/7,eight/8,ninth/9}
  % {\tikzstyle{\x}=[level distance=\pgfmathparse{\y * 6}\pgfmathresult em]};
    % Parents
    \coordinate
    node[fill=blue!10,font=\bf,text width=6cm,draw,text centered] {Client-Server-System für Versicherung}
      % child[grow=left] {node[work package,anchor=east]{Jim}}
      % child[grow=right] {node[work package,anchor=west]{Jane}}
      % child[grow=down,level distance=0ex]
    [edge from parent fork down]
    % Children and grandchildren
    child{node[work package,phase] {Phase 1}
       child[grandchild,first] {node[work package] {\textbf{A:} Funktionen erarbeiten}}
       child[grandchild,second] {node[work package] {\textbf{B:} Funktionen einteilen}}
       child[grandchild,third] {node[work package] {\textbf{C:} Schnittstelllen zu Fremdsystemen}}
       child[grandchild,fourth] {node[work package] {\textbf{D:} Datenbankentwurf}}
       child[grandchild,fifth] {node[work package] {\textbf{E:} GUI-Prototyp entwickeln}}
 }
    child{node[work package,phase] {Phase 2}
       child[grandchild,first] {node[work package] {\textbf{F:} DB-Migration}}
       child[grandchild,second] {node[work package] {\textbf{G:} DB-Test}}
       child[grandchild,third] {node[work package] {\textbf{H:} Schnittstelllen zu Fremdsystemen}}
       child[grandchild,fourth] {node[work package] {\textbf{I:} Clientimplementierung}}
       child[grandchild,fifth] {node[work package] {\textbf{J:} Server-DB-Intergation}}
       child[grandchild,sixth] {node[work package] {\textbf{K:} Client-Server-Integration}}
       child[grandchild,seventh] {node[work package] {\textbf{L:} 1. Systemtest}}
    }
    child{node[work package,phase] {Phase 3}
       child[grandchild,first] {node[work package] {\textbf{M:} Serverfunkt. überarbeiten}}
       child[grandchild,second] {node[work package] {\textbf{N:} Ausbaufunktion.  implementieren}}
       child[grandchild,third] {node[work package] {\textbf{O:} GUI-Fehler beheben}}
       child[grandchild,fourth] {node[work package] {\textbf{P:} GUI für Ausbaufunkt. implementieren}}
       child[grandchild,fifth] {node[work package] {\textbf{Q:} Vollt. Datenmigration}}
       child[grandchild,sixth] {node[work package] {\textbf{R:} Integration Ausbaufunkt.  und Ausbau-GUI}}
       child[grandchild,seventh] {node[work package] {\textbf{S:} 2. Systemtest (Testfiliale)}}
       child[grandchild,eighth] {node[work package] {\textbf{T:} 2. Systemtest (Testteam)}}
    }
    child{node[work package,phase] {Phase 4}
       child[grandchild,first] {node[work package] {\textbf{U:} Korrektur/Optimierung}}
       child[grandchild,second] {node[work package] {\textbf{V:} Doku/Handbuch}}
       child[grandchild,third] {node[work package] {\textbf{W:} Schulung}}
       child[grandchild,fourth] {node[work package] {\textbf{X:} Deployment}}
    };

\end{tikzpicture}
   
\end{center}
\subsection{Vorgangsliste (15 Punkte)}
\begin{center}
   \rowcolors{2}{gray!10}{white}
   \begin{tabular}{llll}
      \toprule
      Bezeichner & Arbeitspaket & Abhängigkeiten & Dauer\\
      \toprule
      \rowcolor{gray!10}
      A & Funktionen erarbeiten & & 2W\\
      B & Funktionen einteilen & A & 1W\\
      C & Schnittstelllen zu Fremdsystemen & A & 1W\\
      D & Datenbankentwurf & & 2W \\
      E & GUI-Prototyp entwickeln & A & 1.5W\\
      F & DB-Migration & D & 2W \\
      G & DB-Test & F& 1W \\
      H & Schnittstelllen zu Fremdsystemen & F, B, C & 7W \\
      I & Clientimplementierung & E & 5W \\
      J & Server-DB-Intergation & H, G& 1W \\
      K & Client-Server-Integration & I, J, & 1W \\
      L & 1. Systemtest & K & 1W \\
      M & Serverfunkt. überarbeiten & L, H, J& 2W \\
      N & Ausbaufunktion.  implementieren & B, H & 4W\\
      O & GUI-Fehler beheben & E, K & 1W \\
      P & GUI für Ausbaufunkt. implementieren & N, O & 2W \\
      Q & Vollt. Datenmigration & G & 2W \\
      R & Integration Ausbaufunkt.  und Ausbau-GUI & P, N, M, O & 2W \\
      S & 2. Systemtest (Testfiliale) & R, Q & 1W \\
      T & 2. Systemtest (Testteam) & R, Q & 1W \\
      U & Korrektur/Optimierung & S, T & 2W \\
      V & Doku/Handbuch & U & 2W \\
      W & Schulung & U, V & 2W \\
      X & Deployment &  V, W & 4W\\
      \bottomrule
   \end{tabular}
\end{center}
\subsection{Netzplan (25 Punkte)}
\subsection{Gantt-Diagramm (20 Punkte)}
\subsection{Firmenorganisation (10 Punkte)}

\end{document}
