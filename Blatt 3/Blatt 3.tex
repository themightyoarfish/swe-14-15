\documentclass{scrartcl}
\title{\rmfamily Software Engineering -- Blatt 3}
\author{Rasmus Diederichsen \and Felix Breuninger\and % ugly hack, why does \\ ... work no more?
   \texttt{\{rdiederichse, fbreunin\}@uos.de}
}
\date{\today}
\usepackage[ngerman]{babel}
\usepackage[table]{xcolor}
\usepackage{arydshln}
\usepackage{microtype,
   textcomp,
   xifthen,
   multirow,
   booktabs,
   dingbat,
   titlesec,
   enumitem,
   fullpage,
   tikz,
   IEEEtrantools,
   array,
   amsmath,
   amssymb,
   graphicx,
   subcaption,
lmodern}
\usepackage[pdftitle={Software Engineering -- Blatt 3}, 
   pdfauthor={Rasmus Diederichsen}, 
   hyperfootnotes=true,
   colorlinks,
   bookmarksnumbered = true,
   linkcolor = lightgray,
   plainpages = false,
citecolor = lightgray]{hyperref}
\usepackage[utf8]{inputenc}
\usepackage[T1]{fontenc}
\usepackage[all]{hypcap}
% \renewcommand{\subsection}[1]{\noindent\bf Aufgabe \arabic{section}.\arabic{subsection}: #1}
\titleformat{\subsection}[hang]{\bf}{Aufgabe \arabic{section}.\arabic{subsection}:}{1em}{}[]
\titleformat{\subsubsection}[hang]{\bf}{\hspace{1em}\alph{subsubsection})}{1em}{}[]


\begin{document}

\fontfamily{ptm}\selectfont
\maketitle


\setcounter{section}{3}
\setcounter{subsection}{0}
\subsection{Projektstrukturplan (20 Punkte)}

Wir wählen einen phasenorientierten PSP. Dies bietet sich an, da der Aufgabentext
diese schon explizit vorgibt, was uns deshalb Arbeit erspart.\leftthumbsup

\usetikzlibrary{trees,calc}
\begin{center}
   \begin{tikzpicture}[
         work package/.style={draw,rectangle,text width=3cm},
         phase/.style={fill=gray!20,rounded corners=5pt,line width=1pt,text centered,font=\bf},
         grandchild/.style={grow=down,
         edge from parent path={(\tikzparentnode.west) -- ++(-1em,0) |- ($(\tikzparentnode.south west) + (-1em,0)$) |- (\tikzchildnode.west)}},
         first/.style={level distance=6ex},
         second/.style={level distance=14ex},
         third/.style={level distance=22ex},
         fourth/.style={level distance=30ex},
         fifth/.style={level distance=38ex},
         sixth/.style={level distance=46ex},
         seventh/.style={level distance=54ex},
         eighth/.style={level distance=62ex},
         level 1/.style={sibling distance=4cm,level distance=2cm}
      ]
      % \foreach \x / \y in {first/1,second/2,third/3,fourth/4,fifth/5,sixth/6,seventh/7,eight/8,ninth/9}
      % {\tikzstyle{\x}=[level distance=\pgfmathparse{\y * 6}\pgfmathresult em]};
      % Parents
      \coordinate
      node[fill=blue!10,font=\bf,text width=6cm,draw,text centered] {Client-Server-System für Versicherung}
      % child[grow=left] {node[work package,anchor=east]{Jim}}
      % child[grow=right] {node[work package,anchor=west]{Jane}}
      % child[grow=down,level distance=0ex]
      [edge from parent fork down]
      % Children and grandchildren
      child{node[work package,phase] {Phase 1}
         child[grandchild,first] {node[work package] {\textbf{A:} Funktionen erarbeiten}}
         child[grandchild,second] {node[work package] {\textbf{B:} Funktionen einteilen}}
         child[grandchild,third] {node[work package] {\textbf{C:} Schnittstellen zu Fremdsystemen}}
         child[grandchild,fourth] {node[work package] {\textbf{D:} Datenbankentwurf}}
         child[grandchild,fifth] {node[work package] {\textbf{E:} GUI-Prototyp entwickeln}}
      }
      child{node[work package,phase] {Phase 2}
         child[grandchild,first] {node[work package] {\textbf{F:} DB-Migration}}
         child[grandchild,second] {node[work package] {\textbf{G:} DB-Test}}
         child[grandchild,third] {node[work package] {\textbf{H:} Schnittstelllen zu Fremdsystemen}}
         child[grandchild,fourth] {node[work package] {\textbf{I:} Clientimplementierung}}
         child[grandchild,fifth] {node[work package] {\textbf{J:} Server-DB-Intergation}}
         child[grandchild,sixth] {node[work package] {\textbf{K:} Client-Server-Integration}}
         child[grandchild,seventh] {node[work package] {\textbf{L:} 1. Systemtest}}
      }
      child{node[work package,phase] {Phase 3}
         child[grandchild,first] {node[work package] {\textbf{M:} Serverfunkt. überarbeiten}}
         child[grandchild,second] {node[work package] {\textbf{N:} Ausbaufunktion.  implementieren}}
         child[grandchild,third] {node[work package] {\textbf{O:} GUI-Fehler beheben}}
         child[grandchild,fourth] {node[work package] {\textbf{P:} GUI für Ausbaufunkt. implementieren}}
         child[grandchild,fifth] {node[work package] {\textbf{Q:} Vollt. Datenmigration}}
         child[grandchild,sixth] {node[work package] {\textbf{R:} Integration Ausbaufunkt.  und Ausbau-GUI}}
         child[grandchild,seventh] {node[work package] {\textbf{S:} 2. Systemtest (Testfiliale)}}
         child[grandchild,eighth] {node[work package] {\textbf{T:} 2. Systemtest (Testteam)}}
      }
      child{node[work package,phase] {Phase 4}
         child[grandchild,first] {node[work package] {\textbf{U:} Korrektur/Optimierung}}
         child[grandchild,second] {node[work package] {\textbf{V:} Doku/Handbuch}}
         child[grandchild,third] {node[work package] {\textbf{W:} Schulung}}
         child[grandchild,fourth] {node[work package] {\textbf{X:} Deployment}}
      };

   \end{tikzpicture}

\end{center}
\subsection{Vorgangsliste (15 Punkte)}
\begin{center}
   \rowcolors{2}{gray!10}{white}
   \renewcommand{\arraystretch}{1.5}
   \begin{tabular}{l:lllr}
      \toprule
      &Bezeichner & Arbeitspaket & Abhängigkeiten & Dauer\\
      \midrule
      \cellcolor{white}&A & Funktionen erarbeiten & & 2W\\
      \cellcolor{white}&B & Funktionen einteilen & A & 1W\\
      \cellcolor{white}&C & Schnittstelllen zu Fremdsystemen & A & 1W\\
      \cellcolor{white}&D & Datenbankentwurf & & 2W \\
      \cellcolor{white}\multirow{-5}{*}{\rotatebox[origin=c]{90}{Phase 1}}
      &E & GUI-Prototyp entwickeln & A & 1.5W\\
      \midrule
      \cellcolor{white}&F & DB-Migration & D & 2W \\
      \cellcolor{white}&G & DB-Test & F& 1W \\
      \cellcolor{white}&H & Schnittstelllen zu Fremdsystemen & F, B, C & 7W \\
      \cellcolor{white}&I & Clientimplementierung & E & 5W \\
      \cellcolor{white}&J & Server-DB-Intergation & H, G& 1W \\
      \cellcolor{white}&K & Client-Server-Integration & I, J & 1W \\
      \cellcolor{white}\multirow{-7}{*}{\rotatebox[origin=c]{90}{Phase 2}}&L & 1. Systemtest & K & 1W \\
      \midrule
      \cellcolor{white}&M & Serverfunkt. überarbeiten & L, H, J& 2W \\
      \cellcolor{white}&N & Ausbaufunktion.  implementieren & B & 4W\\
      \cellcolor{white}&O & GUI-Fehler beheben & E, K & 1W \\
      \cellcolor{white}&P & GUI für Ausbaufunkt. implementieren & H & 2W \\
      \cellcolor{white}&Q & Vollt. Datenmigration & G & 2W \\
      \cellcolor{white}&R & Integration Ausbaufunkt.  und Ausbau-GUI & P, N, M, O & 2W \\
      \cellcolor{white}&S & 2. Systemtest (Testfiliale) & R, Q & 1W \\
      \cellcolor{white}\multirow{-8}{*}{\rotatebox[origin=c]{90}{Phase 3}}&T & 2. Systemtest (Testteam) & R, Q & 1W \\
      \midrule
      \cellcolor{white}&U & Korrektur/Optimierung & S, T & 2W \\
      \cellcolor{white}&V & Doku/Handbuch & U & 2W \\
      \cellcolor{white}&W & Schulung & U, V & 2W \\
      \cellcolor{white}\multirow{-4}{*}{\rotatebox[origin=c]{90}{Phase 4}}&X & Deployment &  V, W & 4W\\
      \bottomrule
   \end{tabular}
\end{center}
\subsection{Netzplan (25 Punkte)}

\usetikzlibrary{positioning,matrix}
% This is for the outer glow. Taken from
% http://tex.stackexchange.com/questions/42810/can-i-have-a-glow-around-a-box-in-tikz
\def\shadowradius{3pt}
%
\newcommand\drawshadowbis[1]{
    \begin{pgfonlayer}{shadow}
        \fill[inner color=red,outer color=white] ($(#1.south west)$) circle (\shadowradius);
        \fill[inner color=red,outer color=white] ($(#1.north west)$) circle (\shadowradius);
        \fill[inner color=red,outer color=white] ($(#1.south east)$) circle (\shadowradius);
        \fill[inner color=red,outer color=white] ($(#1.north east)$) circle (\shadowradius);
        \fill[top color=red,bottom color=white] ($(#1.south west)+((0,-\shadowradius)$) rectangle ($(#1.south east)$);
        \fill[left color=red,right color=white] ($(#1.south east)$) rectangle ($(#1.north east)+((\shadowradius,0)$);
        \fill[bottom color=red,top color=white] ($(#1.north west)$) rectangle ($(#1.north east)+((0,\shadowradius)$);
        \fill[right color=red,left color=white] ($(#1.south west)$) rectangle ($(#1.north west)+(-\shadowradius,0)$);
\end{pgfonlayer}
}
%
\pgfdeclarelayer{shadow} 
\pgfsetlayers{shadow,main}

\tikzstyle{cell}=[minimum height=1.5em,anchor=west,draw,text centered,text width=1.5em]
\newcommand{\boxnode}[5]{% \boxnode{name}{earliest start}{duration}{latest end}{pos cmds}
   \matrix[
      #5,
      matrix of nodes,
      nodes={cell},
      inner sep=0pt,
      outer sep=0pt,
      ampersand replacement=\&,
      row sep=-.5pt,
      column sep=-.5pt,
      nodes={font=\small},
      row 1/.style={nodes={fill=white,draw=none}},
      row 2/.style={nodes={fill=gray!10}},
      row 3/.style={nodes={fill=gray!30}},
   ] (#1) {
      {} \& #1 \& {}\\ % pgfmathparse parses the expression, pgfmathprintnumber is for ommitting the fractional part
      #2\& #3 \& \pgfmathparse{#2+#3}\pgfmathprintnumber{\pgfmathresult}\\
      \pgfmathparse{#4-#3}\pgfmathprintnumber{\pgfmathresult}\& \pgfmathparse{#4-#3-#2}\pgfmathprintnumber{\pgfmathresult} \& #4 \\
   };
   \draw (#1-1-1.south west) -- (#1-1-1.north west) -- (#1-1-3.north east) -- (#1-1-3.south east);
   \pgfmathparse{int(#4-#3-#2)}
   \ifnum\pgfmathresult=0
      \drawshadowbis{#1}
   \fi
}
\usetikzlibrary{arrows}
\tikzset{
arrow/.style={-latex, shorten >=1ex, shorten <=1ex}}

Legende:

\begin{center}
      \begin{tikzpicture}[transform canvas={scale=0.6}]
         \boxnode{A}{0}{2}{5}{}
         \node[left=of A.west] (start) {Frühester Beginn};
         \node[below=of start] (start_late) {Spätester Beginn};
         \node[below=of A.south] (puffer) {Puffer};
         \node[right=of A.east] (end) {Frühestes Ende};
         \node[below=of end] (end_late) {Spätestes Ende};
         \node[above=of A] (bez) {Bezeichner};
         \node[right=of bez] (dauer) {Dauer};

         \draw[arrow, bend left,bend angle=45] (start) to (A.west);
         \draw[arrow, bend right,bend angle=45] (start_late) to (A.south west);
         \draw[arrow,,shorten >=5pt] (puffer) -- (A.south);
         \draw[arrow, bend left,bend angle=45] (end_late) to (A.south east);
         \draw[arrow, bend right,bend angle=45] (end) to (A.east);
         \draw[arrow,,shorten >=8pt] (dauer) -- (A.center);
         \draw[arrow, bend left,bend angle=45] (bez) to (A.north);
      \end{tikzpicture}
\end{center}

\begin{center}
   \begin{tikzpicture}[node distance=1.5cm]
      \boxnode{A}{0}{2}{3}{}
      \boxnode{B}{2}{1}{4}{right=of A}
      \boxnode{C}{2}{1}{4}{below=of B}
      \boxnode{D}{0}{2}{2}{below=of A}
      \boxnode{E}{2}{1.5}{7}{below=of C}

      \boxnode{F}{2}{2}{4}{right=of B}
      \boxnode{G}{4}{1}{11}{right=of F}
      \boxnode{H}{4}{7}{11}{below=of F}
      \boxnode{I}{3.5}{5}{12}{below=of H}
      \boxnode{J}{11}{1}{12}{right=of H}
      \boxnode{K}{12}{1}{13}{below=of J}
      \boxnode{L}{13}{1}{14}{right=of J}

      \boxnode{M}{14}{2}{16}{below left=2cm and 1cm of E}
      \boxnode{N}{3}{4}{16}{right=of M}
      \boxnode{O}{13}{1}{16}{right=of N}
      \boxnode{Q}{5}{2}{17}{right=of O}
      \boxnode{R}{16}{2}{18}{below=of M}
      \boxnode{P}{11}{2}{16}{right=of R}
      \boxnode{S}{18}{1}{19}{right=of P}

      \boxnode{T}{18}{1}{19}{below=2cm of R}
      \boxnode{U}{19}{2}{21}{right=of T}
      \boxnode{V}{21}{2}{23}{right=of U}
      \boxnode{W}{23}{2}{25}{right=of V}
      \boxnode{X}{25}{2}{27}{right=of W}

      \draw[thick,->] (A.east) -- ($(A.east) + (1em,0em)$) -- (B.west);
      \draw[thick,->] ($(A.east) + (1em,0em)$) |- (C.west);
      \draw[thick,->] ($(A.east) + (1em,0em)$) |- (E.west);
      \draw[thick,->] (D.north) |- ($(F.south) + (-.5em,-1em)$) -| ($(F.south) + (-.5em,0em)$);
      \draw[thick,->] (C.east) -- (H.west);
      \draw[thick,-] (B.east) -| ($(H.west) + (-1em,0em)$);
      \draw[thick,->] ($(B.west) + (0em,-1em)$) -- ++(-1em,0em) |- ($(N.west) + (-1em,0em)$) -- (N.west);
      \draw[thick,->] (F.south) -- (H.north);
      \draw[thick,->] (F.east) -- (G.west);
      \draw[thick,->] (H.east) -- (J.west);
      \draw[thick,->] (E.east) -- (I.west);
      \draw[thick,->] (I.east) -- (K.west);
      \draw[thick,->] (K.east) -| ($(L.west) + (-1em,0em)$) |- (L.west);
      \draw[thick,->] (J.south) -- (K.north);
      \draw[thick,->] (G.south) -- (J.north);
      \draw[thick,->] (E.south) |- ($(O.north) + (0em,1em)$) -- (O);
      \draw[thick,-] (K.south) |- ($(O.north) + (0em,1em)$);
      % \draw[thick,->] (B.east) -| ($(N.east) + (1em,0em)$) -- (N.east);
      \draw[thick,-] ($(J.east) + (0em,-1em)$) -| ($(J.east) + (1em,-13em)$) -| (M.north);
      \draw[thick,-] ($(H.east) + (0em,-1em)$) -| ($(H.east) + (1em,-13em)$) -| (M.north);
      \draw[thick,->] ($(L.east) + (0em,-1em)$) -| ($(L.east) + (1em,-13em)$) -| (M.north);
      \draw[thick,->] ($(H.west) + (0em,-.5em)$) -| ($(P.east) + (1.5em,.5em)$) -- ($(P.east) + (0em,.5em)$);
      \draw[thick,->] (G.east) -| ++(10em,-20em) |- (Q.east);
      % \draw[thick,->] (N.south) |- ($(P.north) + (0em,1em)$) -- (P);
      % \draw[thick,-] (O.south) |- ($(P.north) + (0em,1em)$);
      \draw[thick,-] ($(O.south) + (-1em,0em)$) |- ($(R.north) + (0em,1em)$);
      \draw[thick,->] (M.south) -- (R);
      \draw[thick,-] ($(N.south) + (-1em,0em)$) |- ($(R.north) + (0em,1em)$);
      \draw[thick,-] (P.west) -- ++(-1em,0em) |- ($(R.north) + (0em,1em)$);
      \draw[thick,->] (Q.south) |- (S.east);
      \draw[thick,->] ($(R.south) + (1em,0em)$) -- ++(0em,-1em) -| ($(S.south) + (-1em,0em)$);
      \draw[thick,->] (R.south) -- (T.north);
      \draw[thick,-] ($(Q.south) + (1em,0em)$) |- ($(T.north) + (0em,3em)$);
      \draw[thick,->] (T.east) -- (U.west);
      \draw[thick,->] (U.east) -- (V.west);
      \draw[thick,-] ($(U.north) + (1em,0em)$) -- ++(0em,.5em) -| ($(W.west) + (-1em,0em)$);
      \draw[thick,->] (S.west) -| ($(U.east) + (1em,1em)$) -- ($(U.east) + (0em,1em)$);
      \draw[thick,->] (V.east) -- (W.west);
      \draw[thick,-] ($(V.south) + (1em,0em)$) -- ++(0em,-.5em) -| ($(X.west) + (-1em,0em)$);
      \draw[thick,->] (W.east) -- (X.west);
   \end{tikzpicture}
\end{center}
\subsection{Gantt-Diagramm (20 Punkte)}
\subsection{Firmenorganisation (10 Punkte)}

\end{document}
