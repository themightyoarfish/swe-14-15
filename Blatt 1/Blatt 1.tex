\documentclass{scrartcl}
\title{\rmfamily Software Engineering -- Blatt 1}
\author{Rasmus Diederichsen \and Felix Breuninger\and % ugly hack, why does \\ ... work no more?
   \texttt{\{rdiederichse, fbreunin\}$@$uos.de}
}
\date{\today}
\usepackage[ngerman]{babel}
\usepackage{microtype,
            textcomp,
            titlesec,
            tikz-uml,
            fullpage,
            xcolor,
            listings,
            tikz,
            IEEEtrantools,
            array,
            amsmath,
            amssymb,
            graphicx,
            subcaption,
            lmodern}
\usepackage[pdftitle={Software Engineering -- Blatt 1}, 
       pdfauthor={Rasmus Diederichsen}, 
       hyperfootnotes=true,
       colorlinks,
       bookmarksnumbered = true,
       linkcolor = lightgray,
       plainpages = false,
       citecolor = lightgray]{hyperref}
\usepackage[utf8]{inputenc}
\usepackage[T1]{fontenc}
\usepackage[all]{hypcap}
% \renewcommand{\subsection}[1]{\noindent\bf Aufgabe \arabic{section}.\arabic{subsection}: #1}
\titleformat{\subsection}[hang]{\bf}{Aufgabe \arabic{section}.\arabic{subsection}:}{1em}{}[]

\lstset{
   basicstyle=\footnotesize\ttfamily,
   language=Java,
   breaklines=true,
   commentstyle=\color{blue},
   keywordstyle=\color{purple}\textbf,
   numberstyle=\tiny\color{gray},
   numbers=left,
   stringstyle=\color{olive},
}

\begin{document}

\fontfamily{ptm}\selectfont
\maketitle


\setcounter{section}{1}
\setcounter{subsection}{1}
\subsection{Grundlagen: Software-Krise und Software Engineering (40 Punkte)}

\begin{enumerate} 
   \item In den ersten Computersystemen diente Software ausschließlich der
      Berechnung eines einzelnen Algorithmus oder zumindest einer eng begrenzten
      Problemklasse. Durch die schnelle Leistungssteigerung dieser Systeme und
      der Ausweitung der Anwendergruppe von Spezialisten zu gewöhnlichen
      Menschen ohne weiterführende Computer-Kenntnisse stiegen die Anforderungen
      an Software. Somit wurde Software vielseitiger, umfassender und setzte
      komplexere Algorithmen ein. Diesem Wachstum im Rahmen von Umfang, Zahl der
      beteiligten Entwickler und Komponenten hielt der bislang unsystematische
      Entwicklungsvorgang von Software nicht stand. Diese fehlerbehaftete
      Vorgehensweise führte häufig dazu, dass Software nicht zum geplanten Preis
      oder in der geplanten Zeit, teilweise gar nicht, fertig gestellt werden
      konnte. Ausserdem war Software dementsprechend ineffizient, schlecht
      wartbar und nicht portierbar. Diese Fehler in der Funktionalität führten
      zu teuren oder sogar lebensgefährlichen Resultaten.  
   \item Nachdem auf einer Tagung des NATO Science Commitee 1968 diese
      Problematik konstatiert wurde, rückte verstärkt die Entwicklung
      von Werkzeugen und Strategien in den Fokus der Forschung, um diese in
      den Griff zu bekommen. Da die Ingenieursdisziplin viele ähnliche
      Probleme  bereits gelöst hatte, entstand analog dazu das
      Forschungsgebiet des Software Engineering.  Ziel war es, das Vorgehen
      bei der Softwareentwicklung auf ein wissenschaftliches und
      systematisches Fundament zu stellen.  
   \item Die Metrik der \textit{Lines of Code} wirft einige Probleme auf. So
      bleibt zu definieren, welche Art von Zeilen in die Kalkulation eingehen
      (z.B.  Leerzeilen oder Kommentare). Die ermittelten Resultate können für
      ein Programm selber Funktionalität stark variieren abhängig von
      Programmierstil, verwendeter Bibliotheken, (Höhe der) Programmiersprache
      und autogeneriertem Code.  
\end{enumerate}

\subsection{Objektorientierter Entwurf (60 Punkte)}

\usetikzlibrary{calc,positioning,arrows}
\begin{tikzpicture}[every node/.style={font=\footnotesize\sffamily}]
   \tikzumlset{font=\small\sffamily}
   \begin{umlpackage}{Sudoku}

      \umlclass[x=0,y=0]{SudokuRow}{}{}
      \umlclass[x=4,y=0]{SudokuColumn}{}{}
      \umlclass[x=8,y=0]{3x3Field}{}{}
      \umlclass[x=4,y=4]{Sudoku}{}{}
      \umlclass[x=11,y=6]{Player}{}{}
      \umlclass[x=4,y=-4]{Field}{}{}
      \umlclass[x=11,y=-4]{Move}{}{}
      \umlclass[x=4,y=-8]{Number}{}{}

      % dummy node to expand the package box. urg.
      \umlclass[x=14,y=6,draw=none,fill=none,text=blue!20]{I}{}{}

      \umlunicompo[geometry=-|,pos1=.2,pos2=1.9,name=rows,arg2=rows,arg1=board,mult1=1,mult2=9]{Sudoku}{SudokuRow}
      \umlunicompo[pos1=.2,pos2=.9,name=cols,arg2=columns,arg1=board,mult1=1,mult2=9]{Sudoku}{SudokuColumn}
      \umlunicompo[geometry=-|,pos1=.2,pos2=1.9,name=subfields,arg2=subfields,arg1=board,mult1=1,mult2=9]{Sudoku}{3x3Field}
      \umluniassoc[pos1=.1,pos2=1.8,arg1=player,arg2=game,name=plays,geometry=-|,mult2=0..*,mult1=1]{Player}{Sudoku}
      % \umlbiassoc[geometry=-|,anchors=215 and 145,arm1=1.5cm]{Player}{Sudoku}
      \umluniaggreg[geometry=|-,anchors=275 and 180,arg1=row,arg2=fields,pos2=1.8,name=rowFields,mult1=1,mult2=9]{SudokuRow}{Field}
      \umluniaggreg[arg1=column,arg2=fields,name=colFields,mult1=1,mult2=9]{SudokuColumn}{Field}
      \umluniaggreg[geometry=-|-,anchors=200 and 40,pos1=.6,pos2=1.9,arg1=subfield,arg2=fields,name=subFields,mult1=1,mult2=9]{3x3Field}{Field}
      \umluniaggreg[arg1=field,arg2=contained number,mult1=*,mult2=1,name=contains]{Field}{Number}
      \umluniassoc[mult2=1,mult1=*,arg2=chosen number,arg1=move,name=choose,geometry=|-,pos2=1.8,anchors=275 and 0]{Move}{Number}
      \umluniassoc[pos1=.1,pos2=.9,arg1=move,mult2=1,arg2=field,mult1=0..*,name=setFieldTo]{Move}{Field}
      \umluniassoc[arg1=last,mult2=0..1,arg2=previous,mult1=0..1,pos1=.05,name=precursor,pos2=.9,angle1=45,angle2=0,loopsize=3cm]{Move}{Move}
      \umluniaggreg[arg1=player,mult2=*,arg2=move,mult1=1,name=makeMove]{Player}{Move}
      \umluniassoc[geometry=|-,anchors=135 and
      40,arg1=move,arg2=board,mult1=1,mult2=*,name=makeMoveOn,pos2=1.7,pos1=.1]{Move}{Sudoku}

      \node () at (rows-2) [fill=blue!20,text opacity=1] {$\triangleleft$ consists of};
      \node () at (cols-1) [fill=blue!20,text opacity=1] {$\triangledown$ consists of};
      \node () at (subfields-2) [fill=blue!20,text opacity=1] {$\triangleright$ consists of};
      \node () at (subFields-3) [fill=blue!20,text opacity=1] {$\triangledown$ consists of};
      \node () at (plays-2) [fill=blue!20,text opacity=1] {$\triangleleft$ plays}; 
      \node () at (rowFields-2) [fill=blue!20,text opacity=1] {$\triangleright$ consists of}; 
      \node () at (colFields-1) [fill=blue!20,text opacity=1] {$\triangledown$ consists of}; 
      \node () at (contains-1) [fill=blue!20,text opacity=1] {$\triangledown$ contains}; 
      \node () at (choose-2) [fill=blue!20,text opacity=1] {$\triangleleft$ choose}; 
      \node () at ($(precursor-1) + (2pt,0)$) [fill=blue!20,text opacity=1] {$\triangledown$ precursor}; 
      \node () at ($(makeMove-1) + (.5cm,0)$) [fill=blue!20,text opacity=1] {$\triangledown$ make move}; 
      \node () at ($(makeMoveOn-3) + (0,0cm)$) [fill=blue!20,text opacity=1] {$\triangleleft$ on}; 

      \umlnote[y=-6.5,x=.5]{Number}{0-9, 0 := free}
   \end{umlpackage}
\end{tikzpicture}
\end{document}
