\documentclass{scrartcl}
\title{\rmfamily Software Engineering -- Blatt 5}
\author{Rasmus Diederichsen \and Felix Breuninger\and 
   \texttt{\{rdiederichse, fbreunin\}@uos.de}
}
\date{\today}
\usepackage[ngerman]{babel}
\usepackage[table]{xcolor}
\usepackage{arydshln}
\usepackage{marvosym, microtype, textcomp, xifthen, multirow, booktabs, dingbat,
   titlesec, enumitem, fullpage, tikz, IEEEtrantools, array, amsmath, listings,
amssymb, graphicx, subcaption, lmodern,pgfplots}
\usepackage[pdftitle={Software Engineering -- Blatt 5}, 
   pdfauthor={Rasmus Diederichsen, Felix Breuninger}, 
   hyperfootnotes=true,
   colorlinks,
   bookmarksnumbered = true,
   linkcolor = lightgray,
   plainpages = false,
citecolor = lightgray]{hyperref}
\usepackage[utf8]{inputenc}
\usepackage[T1]{fontenc}
\usepackage[all]{hypcap}
\titleformat{\subsection}[hang]{\bf}{Aufgabe \arabic{section}.\arabic{subsection}:}{1em}{}[]
\titleformat{\subsubsection}[hang]{\bf}{\hspace{1em}\alph{subsubsection})}{1em}{}[]

\lstset{
   frame=single,
   basicstyle=\ttfamily,
   frameround=tttt,
   backgroundcolor=\color{lightgray!10}
}
\begin{document}

\fontfamily{ptm}\selectfont
\maketitle

\setcounter{section}{5}
\setcounter{subsection}{0}

\subsection{SVN und GIT}

\subsubsection{SVN}

Ein neues leeres Repository wird serverseitig mit 
\begin{lstlisting}
svnadmin create <directory>
\end{lstlisting} angelegt. Der lokale Benutzer kann mit 
\begin{lstlisting}
svn checkout <url> <directory>
\end{lstlisting} das hinter der \texttt{url}
 befindliche Repository in das lokale Verzeichnis 
\texttt{dir} klonen. Damit werden alle im Repository enthaltenen Dateien in das
Verzeichnis kopiert. Hat man im lokalen Verzeichnis neue Dateien erzeugt, kann
man diese per 
\begin{lstlisting}
svn add <file1> <file2> ...
\end{lstlisting} dem lokalen Repository hinzufügen. Mit 
\begin{lstlisting}
svn commit -m <message>
\end{lstlisting} werden die Änderungen
(Hinzugefügtes, Entferntes, Geändertes) auf den Server übertragen. Möchte man
eine lokale Änderung rückgängig machen, so kann man dies vor dem Commit mit

\begin{lstlisting}
svn revert <file>
\end{lstlisting}
 tun. Die Datei wird dann auf den Status nach dem
letzten \texttt{update} zurückgesetzt. Möchte man Dateien umbenennen oder
Verschieben, muss man 
\texttt{svn} darüber in Kenntnis setzen und muss den
Betriebssystemkommandos 
\texttt{mkdir}, \texttt{mv}, \texttt{cp} und 
\texttt{rm} ein `` \texttt{svn}''
voranstellen, um die Änderungen auch dem Repository mitzuteilen. Beim nächsten
Commit werden diese dann auf den Server übertragen.

Will man einen neuen Branch erzeugen, kann man dies direkt auf dem Server tun,
ohne selbst eine Arbeitskopie zu besitzen. Für SVN ist ein Branch einfach nur
eine Kopie eines Verzeichnisbaumes. Per Konvention enthält jedes Projekt die
Unterverzeichnisse \texttt{trunk} (Hauptentwicklungspfad), \texttt{branches}
(Verzweigungen) und \texttt{ tags } (stabile Snapshots). Neue Branches werden in
\texttt{ branches } angelegt, z.B. durch 
\begin{lstlisting}
svn copy file://<path to repo>/<project name>/trunk  \
         file://<path to repo>/<project name>/branches/dev \
         -m ``<message>''
\end{lstlisting} . Hierbei wird ein neuer Branch 
\texttt{dev} auf dem Server erstellt. Nach einem \texttt{update} kann man nun
darin arbeiten. Will man den Branch wieder in den \texttt{trunk} mergen,
navigiert man in ihn hinein (in der lokalen Kopie) und führt
\begin{lstlisting}
svn merge ^/<project name>/trunk
\end{lstlisting} aus, wobei ``\^{}'' für die URL des Repos steht. 
Man merged einen Branch \texttt{b1} in einen Branch \texttt{b2}, indem man sich
im Verzeichnis \texttt{b2} befindet und 
\begin{lstlisting}
svn merge ^/<project name>/b1
\end{lstlisting}
aufruft. Äderungen in \texttt{b1} werden so in \texttt{b2} übertragen. Will man
einen privaten Branch wieder in den \texttt{trunk} mergen, ruft man
\begin{lstlisting}
svn merge --reintegrate ^/<project name>[/branches]/<private branch>
\end{lstlisting} aus dem \texttt{trunk}-Verzeichnis auf.
Wurden an derselben Datei in beiden Branches unvereinbare Änderungen
durchgeführt, so müssen diese aufgelöst werden, z.B. manuell. Nachdem man
Konflikte bereinigt hat, kann man die Änderungen \texttt{commit}ten und den
obsoleten privaten Branch löschen.
\end{document}
