\documentclass{scrartcl}
\title{\rmfamily Software Engineering -- Blatt 4}
\author{Rasmus Diederichsen \and Felix Breuninger\and % ugly hack, why does \\ ... work no more?
   \texttt{\{rdiederichse, fbreunin\}@uos.de}
}
\date{\today}
\usepackage[ngerman]{babel}
\usepackage[table]{xcolor}
\usepackage{arydshln}
\usepackage{microtype, textcomp, xifthen, multirow, booktabs, dingbat,
   titlesec, enumitem, fullpage, tikz, IEEEtrantools, array, amsmath,
   amssymb, graphicx, subcaption, lmodern}
\usepackage[pdftitle={Software Engineering -- Blatt 4}, 
   pdfauthor={Rasmus Diederichsen, Felix Breuninger}, 
   hyperfootnotes=true,
   colorlinks,
   bookmarksnumbered = true,
   linkcolor = lightgray,
   plainpages = false,
citecolor = lightgray]{hyperref}
\usepackage[utf8]{inputenc}
\usepackage[T1]{fontenc}
\usepackage[all]{hypcap}
% \renewcommand{\subsection}[1]{\noindent\bf Aufgabe \arabic{section}.\arabic{subsection}: #1}
\titleformat{\subsection}[hang]{\bf}{Aufgabe \arabic{section}.\arabic{subsection}:}{1em}{}[]
\titleformat{\subsubsection}[hang]{\bf}{\hspace{1em}\alph{subsubsection})}{1em}{}[]


\begin{document}

\fontfamily{ptm}\selectfont
\maketitle

\setcounter{section}{4}
\setcounter{subsection}{0}

\subsection{Projektmanagement-Werkzeuge (15 Punkte)}

\subsection{Projekt-Ressourcen (20 Punkte)}

\subsection{Brook'sches Gesetz (15 Punkte)}

Es gilt für den Kommunikationsaufwand
\begin{equation*}
   k = 2n \cdot {n \choose 2} = 2 \frac{n!}{2(n-2)!} = n(n-1)
\end{equation*}
 Der Gesamtaufwand ist gegeben durch
 \begin{equation*}
    f_E(n) = k \frac{600}{n} = n^2 - n + \frac{600}{n}
 \end{equation*}
 Die Ableitung
 \begin{equation*}
    f_E^\prime(n) = 2n -1 - 600n^{-2}
 \end{equation*}
 besitzt als einzige positive reelle Nullstelle $n_0 \approx 6.8652$.

Der Aufwand $f_E(n)$ bei 3 Mitarbeitern beträgt 206 Stunden pro Mitarbeiter, was
26 Tagen entspricht (bei 8 Stunden täglicher Arbeit).
Bei 7 Mitarbeitern verringert er sich auf $f_E(7)\approx 127,11$ Stunden $\le
16$ Tage pro Mitarbeiter. Durch die Aufstockung hat man also 10 Tage gespart.

\subsection{COCOMO (15 Punkte)}

\subsection{Function-Points (25 Punkte)}

\end{document}
