\documentclass{scrartcl}
\title{\rmfamily Software Engineering -- Blatt 4}
\author{Rasmus Diederichsen \and Felix Breuninger\and % ugly hack, why does \\ ... work no more?
   \texttt{\{rdiederichse, fbreunin\}@uos.de}
}
\date{\today}
\usepackage[ngerman]{babel}
\usepackage[table]{xcolor}
\usepackage{arydshln}
\usepackage{marvosym, microtype, textcomp, xifthen, multirow, booktabs, dingbat,
   titlesec, enumitem, fullpage, tikz, IEEEtrantools, array, amsmath,
amssymb, graphicx, subcaption, lmodern,pgfplots}
\usepackage[pdftitle={Software Engineering -- Blatt 4}, 
   pdfauthor={Rasmus Diederichsen, Felix Breuninger}, 
   hyperfootnotes=true,
   colorlinks,
   bookmarksnumbered = true,
   linkcolor = lightgray,
   plainpages = false,
citecolor = lightgray]{hyperref}
\usepackage[utf8]{inputenc}
\usepackage[T1]{fontenc}
\usepackage[all]{hypcap}
% \renewcommand{\subsection}[1]{\noindent\bf Aufgabe \arabic{section}.\arabic{subsection}: #1}
\titleformat{\subsection}[hang]{\bf}{Aufgabe \arabic{section}.\arabic{subsection}:}{1em}{}[]
\titleformat{\subsubsection}[hang]{\bf}{\hspace{1em}\alph{subsubsection})}{1em}{}[]


\begin{document}

\fontfamily{ptm}\selectfont
\maketitle

\setcounter{section}{4}
\setcounter{subsection}{0}

\subsection{Projektmanagement-Werkzeuge (15 Punkte)}
siehe Projektdatei

\subsection{Projekt-Ressourcen (20 Punkte)}
\subsubsection{}
siehe Projektdatei
\subsubsection{}
siehe Projektdatei

\subsubsection{}
Durch Bilbos Urlaub verschiebt sich die Erledigung des Vorgangs Implementation der Ausbaufunktionalität um den entsprechenden Zeitraum. Da dieser Vorgang auf keinem kritischen Pfad liegt, ist dies die einzige Konsequenz und hat keine weiteren Auswirkungen auf das Gesamtprojekt.

\subsubsection{}
Da Fili Mitglied des GUI-Teams ist und deren Arbeit erst ab dem 14.12. beginnt, hat Filis Urlaub keine Auswirkungen auf das Projekt oder Vorgänge des Projektes.

\subsection{Brook'sches Gesetz (15 Punkte)}

Es gilt für den Kommunikationsaufwand
\begin{equation*}
   k = 2n \cdot {n \choose 2} = 2 \frac{n!}{2(n-2)!} = n(n-1)
\end{equation*}
Der Gesamtaufwand ist gegeben durch
\begin{equation*}
   f_E(n) = k \frac{600}{n} = n^2 - n + \frac{600}{n}
\end{equation*}
Die Ableitung
\begin{equation*}
   f_E^\prime(n) = 2n -1 - 600n^{-2}
\end{equation*}
besitzt als einzige positive reelle Nullstelle $n_0 \approx 6.8652$.

Der Aufwand $f_E(n)$ bei 3 Mitarbeitern beträgt 206 Stunden pro Mitarbeiter, was
26 Tagen entspricht (bei 8 Stunden täglicher Arbeit).
Bei 7 Mitarbeitern verringert er sich auf $f_E(7)\approx 127,11$ Stunden $\le
16$ Tage pro Mitarbeiter. Durch die Aufstockung hat man also 10 Tage gespart.

\subsection{COCOMO (15 Punkte)}

\subsubsection{}

Da es sich um ein schweres Projekt handelt, beträgt die voraussichtliche Dauer

\begin{equation*}
   V\kern-3ptD = \frac{1000}{350} \cdot 17^{1.28} \approx 107 \text{PM}
\end{equation*}

\subsubsection{}

Hier handelt es sich um ein leichtes Projekt, daher
\begin{equation*}
   V\kern-3ptD = \frac{1000}{450} \cdot 1.5^{1.04} \cdot 1.05 \cdot 0.9 \approx 3.2 \text{PM}
\end{equation*}

\subsection{Function-Points (25 Punkte)}

\subsubsection{}
Zunächst werden die Funktionen kategorisiert und auf ungewichtete Function Points
abgebildet.

\begin{itemize}[font=\textbf,align=left]
   \item[Kontakte eingeben:] Mittelschwere Eingabe $\Rightarrow$ 4
   \item[Kontakte verwalen:] Einfache Datenbestandsfunktion $\Rightarrow$ 7
   \item[Kontakte anzeigen:] Schwierige Abfrage $\Rightarrow$ 6
   \item[Kontakte analysieren:] Schwierige Ausgabe $\Rightarrow$ 7
\end{itemize}

Die ungewichteten FPs sind also
\begin{equation*}
   FP_{\text{ungew}} = 4 + 7 + 6 +7 = 24
\end{equation*}

Die Einflussfaktoren bestehen in
\begin{itemize}[font=\textbf,align=left]
   \item[Komplexe Berechnungen:] Starker Einflussfaktor $\Rightarrow$ 4
   \item[Verschiedene Plattformen:] Starker Einflussfaktor $\Rightarrow$ 4
   \item[End-User-Effizienz:] Kritischer Einflussfaktor $\Rightarrow$ 4
   \item[Weitere Faktoren:] 3-mal stark, 2-mal mittel, 2-mal schwach
      $\Rightarrow 3\cdot 4 + 2\cdot 3 + 2\cdot 2 = 22$
\end{itemize}

Die gewichteten Function Points errechnen sich zu 
\begin{IEEEeqnarray*}{rCl}
   FP_{\text{gew}} & = & FP_{\text{ungew}} \cdot ((4+4+5+22)\cdot .01 + .65) \\
                   & = & 24 \cdot (.35 + .65) \\
                   & = & 24
\end{IEEEeqnarray*}

Als Aufwandsschätzung ergibt sich $24\cdot 960\text{\EUR}=23040\text{\EUR} $.

\subsubsection{}

Laut Übung (glaube ich), ist der Zusammenhang zwischen FPs und PMs ungefähr
exponentiell. Je mehr Funktionen, desto mehr Aufwand muss für die Integration
der Teile veranschlagt werden, zumindest in den meisten Projekten, weshalb der
Zusammenhang nicht linear ist.
\begin{center}
   \begin{tikzpicture}
      \begin{axis}[
            axis lines=left,
            scaled ticks=false,
            xmin=0,
            xmax=6
         ]
         \addplot[black] {exp(x/3)};
      \end{axis}
   \end{tikzpicture}
\end{center}

Die einzige Voraussetzung für die Sinnhaftigkeit der FP-Analyse ist das
Vorhandensein von Erfahrungswerten für die Abbildung $FP \mapsto PM$, wenn man
die Funktionskategorien und deren Bewertung als sinnvoll voraussetzt.

\end{document}
