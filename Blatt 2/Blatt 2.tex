\documentclass{scrartcl}
\title{\rmfamily Software Engineering -- Blatt 2}
\author{Rasmus Diederichsen \and Felix Breuninger\and % ugly hack, why does \\ ... work no more?
   \texttt{\{rdiederichse, fbreunin\}@uos.de}
}
\date{\today}
\usepackage[ngerman]{babel}
\usepackage{microtype,
            textcomp,
            titlesec,
            enumitem,
            fullpage,
            xcolor,
            listings,
            tikz,
            IEEEtrantools,
            array,
            amsmath,
            amssymb,
            graphicx,
            subcaption,
            lmodern}
\usepackage[pdftitle={Software Engineering -- Blatt 2}, 
       pdfauthor={Rasmus Diederichsen}, 
       hyperfootnotes=true,
       colorlinks,
       bookmarksnumbered = true,
       linkcolor = lightgray,
       plainpages = false,
       citecolor = lightgray]{hyperref}
\usepackage[utf8]{inputenc}
\usepackage[T1]{fontenc}
\usepackage[all]{hypcap}
% \renewcommand{\subsection}[1]{\noindent\bf Aufgabe \arabic{section}.\arabic{subsection}: #1}
\titleformat{\subsection}[hang]{\bf}{Aufgabe \arabic{section}.\arabic{subsection}:}{1em}{}[]
\titleformat{\subsubsection}[hang]{\bf}{\hspace{1em}\alph{subsubsection})}{1em}{}[]

\lstset{
   basicstyle=\footnotesize\ttfamily,
   language=Java,
   breaklines=true,
   commentstyle=\color{blue},
   keywordstyle=\color{purple}\textbf,
   numberstyle=\tiny\color{gray},
   numbers=left,
   stringstyle=\color{olive},
}

\begin{document}

\fontfamily{ptm}\selectfont
\maketitle


\setcounter{section}{2}
\setcounter{subsection}{0}
\subsection{Phasen der Software-Entwicklung (16 Punkte)}
\subsection{Modularisierung (30 Punkte)}


\subsubsection{Bindung}

\begin{itemize}[font=\ttfamily,leftmargin=2cm,align=left]
   \item[FieldController] Funktional da der Controller nur eine Funktion
      exportiert.
   \item[GameBoard] Logisch, da zusammenhängende Funktionen angeboten werden,
      die aber nur teilweise auf den selben Daten operieren, sodass keine
      wirkliche kommunikatorische Bindung vorliegt.
   \item[Field] Logisch, da die Operationen nicht alle auf derselben Struktur
      operieren; das \texttt{Field} selbst ist eher die Struktur, auf der
      Operiert wird und ist so nicht von anderne Typen abhängig. Die Klasse
      stellt somit eine Sammlung zusammengehörender Funtkionen dar.
   \item[GameView] Funktional, da es nur eine Funktion (\texttt{update()}) gibt.
   \item[FieldButton] Wäre \texttt{refreshView()} private, so wäre das Modul
      funktional gebunden. So jedoch lediglich temporal oder sequentiell, da
      \texttt{update()} immer \texttt{refreshView()} aufruft. Da dies die
      einzige andere Methode ist, lässt sich dies aber schwer beurteilen.
   \item[MineSweeper] Zufällig, da die enthaltenen Funktionen nichts miteinander
      zu tun haben. Betrachtet man nur die exportierte Funktionalität, kann man
      von funktionaler Bindung sprechen.
   \item[GameOverListener] Funktional, da nur eine Funktion vorhanden ist.
\end{itemize}

\subsubsection{Kopplung}

\begin{itemize}[font=\ttfamily,leftmargin=2cm,align=left]
   \item[FieldController] ~% something must be here, else line is not broken
      \begin{itemize}[font=\ttfamily,align=left]
         \item[MouseAdapter] Nicht gekoppelt, da keine Aufrufe an die
            Superklasse außer im Konstruktor
      \end{itemize}
   \item[MineSweeper] ~ 
      \begin{itemize}[font=\ttfamily,align=left]
         \item[GameView] Stempelkopplung, da im Konstruktor der Listener
            übergeben wird, den beide kennen müssen.
         \item[GameBoard] Datenkopplung (Zeile 68)
      \end{itemize}
   \item[GameView] ~
      \begin{itemize}[font=\ttfamily,align=left]
         \item[GameBoard] Stempelkopplung, wegen des
            Konstruktoraufrufs mit einem komplexen Typ (\texttt{GridLayout})
            sowie des Methodenaufrufs in Zeile 79.
         \item[FieldButton] Ebenfalls Stempelkopplung, ebenfalls durch Methode
            in Zeile 79. Der Typ \texttt{Field} muss beiden bekannt sein.
         \item[GameOverListener] Keine direkte Kopplung, da die Aufrufe keine
            Argumente enthalten.
      \end{itemize}
   \item[FieldButton] ~
      \begin{itemize}[font=\ttfamily,align=left]
         \item[Field] Stempelkopplung aufgrund des
            \texttt{addObserver(this)}-Aufrufs
      \end{itemize}
   \item[GameBoard] ~
      \begin{itemize}[font=\ttfamily,align=left]
         \item[Field] Steuerkopplung wegen des Aufrufs \texttt{new Field(true)},
            wobei der Wert eher ein Datum als einen steuernden Schalter
            darstellt, daher kann man auch als Datenkopplung kategorisieren.
      \end{itemize}
   \item[Field] ~
      \begin{itemize}[font=\ttfamily,align=left]
         \item[GameBoard] Datenkopplung wegen des Aufrufs in Zeile 403,
            Stempelkopplung wegen des Aufrufs in Zeile 434.
      \end{itemize}
\end{itemize}
\subsection{Design by Contract, C4J (34 Punkte)}

\end{document}
